%\lipsum[5]

To promote and contribute to more liveable spaces, decision-makers must act on the need to provide more long-term social-inclusive transport solutions. Colombia's capital, Bogotá, has experienced a rapid urbanization process that doubled its population in 30 years. By 2021, the city started constructing the first metro line, expected to be functioning by 2028, together with Bogotá's current bus rapid transit (BRT) infrastructure and transit operation. This study simulated a future transit scenario using an accessibility measure to assess how the construction of the metro line would contribute to making opportunities in Bogotá more reachable. By simulating transit operation using general transport feed specification (GTFS) data, a cumulative accessibility measure was modelled to assess the ease with which Bogotá's inhabitants could reach opportunities using the future transport system. When comparing the current and future scenarios, the results showed the overall and geographically disaggregated accessibility impact, and the locations that would experience the highest gains classified by their different median income levels. The study concludes that a cumulative accessibility measure provides a more integral, social-inclusive and sustainability-oriented assessment as it holistically combines transport infrastructure, transit operation data and spatial distribution of opportunities into one single measure.


