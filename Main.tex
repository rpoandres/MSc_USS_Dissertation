% Author: Joseph Rowell
% Version: 3.0
% This work is licensed under a Creative Commons Attribution 4.0 International License.
\documentclass[12pt, a4paper]{report}
\usepackage[a4paper, total={6.25in, 8.25in}]{geometry} %%%%%% CHECK MARGIN REQ. 8.25 in?
\input{Packages.tex}
\usepackage{multicol}
\hypersetup{pdftitle = Project Report, pdfauthor = {First Last}, pdfstartview=FitH, pdfkeywords = essay, pdfpagemode = FullScreen, colorlinks, anchorcolor = black, citecolor = black, urlcolor = blue, filecolor = green, linkcolor = blue, plainpages = false}
%%%%%%%%%%%%%%%%%%%%%%%%%%%%%%%%%%%%%%%%%%%%%%%%%%%%%%%%%%%%%%%%%%%%%%%
%\pagestyle{fancy}
\rhead{}
\chead{}
\lhead{University College London}
\lfoot{\date{}}
\cfoot{}
\rfoot{\thepage}
% Top and Bottom Line Rules

\renewcommand{\headrulewidth}{0.4pt} %0.4pt
\renewcommand{\footrulewidth}{0.4pt}
\fancyheadoffset{8pt}
\fancyfootoffset{8pt}
% Line spacing
\renewcommand{\baselinestretch}{1.5} %1.5

\makeglossaries

\newglossaryentry{latex}
{
    name=Latex,
    text=latex,
    description={Is a markup language specially suited 
    for scientific documents, this term is printed in conclusion }
}
\newglossaryentry{raster}
{
    name=Raster,
    text=raster,
    description={ images are compiled using pixels, or tiny dots, containing unique color and tonal information that come together to create the image }
}
\newglossaryentry{gradient descent}
{
    name=Gradient descent,
    text=gradient descent,
    description={Is a naive optimization method which consists of steepest descent down the gradient of the given cost function}
}

\newglossaryentry{Gauss-Newton}
{
    name=Gauss-Newton,
    description={Is a Newton-like method for solving a non-linear least squares problem, in which the Hessian $H$ is approximated by $H \approx J^T WJ$, where $J$ is the design matrix and $W$ is the weights. The normal equations are the resulting prediction equations given as \\ $(J^TWJ) \delta x = -(JW \Delta z)$}
}

\newglossaryentry{Conjugate Gradient}
{
    name=Conjugate Gradient,
    text=conjugate gradient,
    description={Is an accelerated first order iterative process for solving positive definite linear systems or minimizing a non linear cost function.}
}

\newglossaryentry{Jacobian}
{
    name=Jacobian,
    description={Matrix of partial differentials of the cost function $J = \frac{df}{dx}$}
}

\newglossaryentry{Hessian}
{
    name=Hessian,
    description={Matrix of second partial differentials of the cost function $H = \frac{d^2f}{dx^2}$}
}
\newglossaryentry{Gradient}
{
    name=Gradient,
    description={First differential $g = \frac{df}{dx}$}
}
\newglossaryentry{epipolar plane}
{
    name=Epipolar Plane,
    description={The plane containing the intersection line joining the camera centres with the image plane.}
}
\newglossaryentry{linear least squares}
{ 
    name =Linear Least Squares,
    description ={Least squares approximation of linear functions to data, by minimizing residuals.
     $E_{LS} = \sum_i{||\hat{x'_i}- \tilde{x'_i}||}$}
}
\newglossaryentry{RANdom Sample Consensus (RANSAC)}
{
    name = Random Sample Consensus (RANSAC),
    description = {An iterative method to estimate parameters of a mathematical model from a set of observed data which contains outliers.}
}


\date{August 2023}

\title{Accessibility impact of transport infrastructure: Spatial assessment of Bogot\'{a}\textquotesingle s future metro system}
\author{\\ \Large{Andr\'{e}s Restrepo Jim\'{e}nez}
\\ Supervisor: Dr. Fulvio Lopane 
\\ Module: CASA0010
%\\ The Bartlett Faculty of the Built Environment
%\\ Centre for Advanced Spatial Analysis
\\
\\ Remote repo link
\\ Word count: XXXXX
\\
%\\ University College London
\\
This dissertation is submitted in part requirement for \\the \textit{MSc Urban Spatial Science} in the \\Centre for Advanced Spatial Analysis, \\Bartlett Faculty of The Build Environment, \\University College London.
\\ \\
% \\Disclaimer:This report is submitted as part requirement for the MSc in Robotics  and Computation at University College London. It is substantially the result of my own work except where explicitly indicated in the text.The report may be freely copied and distributed provided the source is explicitly acknowledged.
% September 2022
}
%\hypersetup{citecolor=black}
%%%%%%%%%%%%%%%%%%%%%%%%%%%%%%%%%%%%%%%%%%%%%%%%%%%%%%%%%%%%%%%%%%%%%%%
\begin{document}
%Adjust logo positions here
% \AddToShipoutPicture*{\parbox[t][\paperheight][t]{\paperwidth}{%
%           \includegraphics[width=\paperwidth]{\BackgroundPicturea{Logos/ucl_long%_logo.png}{3in}{3in}}
%           }}
% \AddToShipoutPicture*{\centering\BackgroundPictureb{Logos/Bentham2011_065_c623d.jpg}{3in}{3.7in}}
\AddToShipoutPictureBG*{%
  \AtPageUpperLeft{%
    \raisebox{-\height}{%
      \includegraphics[width=\paperwidth]{Logos/UCL page header_V1.png}%
    }}
}
\AddToShipoutPicture*{%
      \parbox[t][\paperheight][t]{\paperwidth}{%
          \includegraphics[width=1.2\paperwidth]{Logos/UCL page footer.png}
      }}
      

\thispagestyle{headings}
\maketitle
\FloatBarrier
\pagenumbering{roman}

\thispagestyle{empty}
\begin{abstract}
%\lipsum[5]

To promote and contribute to more liveable spaces, decision-makers must act on the need to provide more long-term social-inclusive transport solutions. Colombia's capital, Bogotá, has experienced a rapid urbanization process that doubled its population in 30 years. By 2021, the city started constructing the first metro line, expected to be functioning by 2028, together with Bogotá's current bus rapid transit (BRT) infrastructure and transit operation. This study simulated a future transit scenario using an accessibility measure to assess how the construction of the metro line would contribute to making opportunities in Bogotá more reachable. By simulating transit operation using general transport feed specification (GTFS) data, a cumulative accessibility measure was modelled to assess the ease with which Bogotá's inhabitants could reach opportunities using the future transport system. When comparing the current and future scenarios, the results showed the overall and geographically disaggregated accessibility impact, and the locations that would experience the highest gains classified by their different median income levels. The study concludes that a cumulative accessibility measure provides a more integral, social-inclusive and sustainability-oriented assessment as it holistically combines transport infrastructure, transit operation data and spatial distribution of opportunities into one single measure.




\keywords{keyword 1 - keyword 2 - keyword 3}
% \vspace{-10mm} %To remove added white space after
\end{abstract}
\newpage
\thispagestyle{empty}
\begin{center}
I dedicate this ...
\end{center}

\newpage
\thispagestyle{empty}
\vspace*{\fill}
\begin{center}
Copyright \copyright  \thinspace 2023 by Andr\'{e}s Restrepo Jim\'{e}nez \\ All Rights Reserved
\end{center}
\vspace*{\fill}
\newpage
\thispagestyle{empty}
%\epigraph{The Book of Nature is written in the language of mathematics.}{--- \textup{Galileo Galilei}}
\epigraph{The computer was born to solve problems that did not exist before.}{--- \textup{Bill Gates}}


\thispagestyle{empty}
\chapter*{Acknowledgements}


\thispagestyle{empty}
\chapter*{Declaration}
I, Andr\'{e}s Restrepo Jim\'{e}nez, hereby declare that this dissertation is all my own original work and that all sources have been acknowledged. It is xxx words in length. \\
Add signature png here.
\begin{figure}[H]
\includegraphics{Logos/Signature.jpg}
\end{figure}
\vspace{-2cm}
\noindent\begin{tabular}{ll}
 & 20/08/23 \\
\makebox[2.5in]{\hrulefill} & \makebox[2.5in]{\hrulefill}\\
\textit{Signature} & \textit{Date}\\
\end{tabular}


\tableofcontents
\pagenumbering{arabic}
\thispagestyle{plain}
\listoffigures
\listoftables
%\lstlistoflistings
%\listofalgorithms


%\begin{singlespace}
\chapter*{List of acronyms or abbreviations}
\begin{sortedlist} %sort alphabetically
  \sortitem{BRT: Bus rapid transit}
  \sortitem{GIS: Geopraphic information system}
  \sortitem{GTFS: General transit feed specification}
  \sortitem{GDP: Gross domestic product}
  \end{sortedlist}
% \end{singlespace}
%%%%%%%%%%%%%%%%%%%%%%%%%%%%%%%%%%%%%%%%%%%%%%%%%%%%%%%%%%%%%%%%%%%%%%%%%%%%%%%%
%\input{1 Introduction}
%%%%%%%%%%%%%%%%%%%%%%%%%%%%%%%%%%%%%%%%%%%%%%%%%%%%%%%%%%%%%%%%%%%%%%%%%%%%%%%%
%\input{2 Literature review}
%%%%%%%%%%%%%%%%%%%%%%%%%%%%%%%%%%%%%%%%%%%%%%%%%%%%%%%%%%%%%%%%%%%%%%%%%%%%%%%%
%\input{3 Study area and or data chapter}
%%%%%%%%%%%%%%%%%%%%%%%%%%%%%%%%%%%%%%%%%%%%%%%%%%%%%%%%%%%%%%%%%%%%%%%%%%%%%%%%
%\input{4 Methodology}
%%%%%%%%%%%%%%%%%%%%%%%%%%%%%%%%%%%%%%%%%%%%%%%%%%%%%%%%%%%%%%%%%%%%%%%%%%%%%%%%
%\input{5 Results}
%%%%%%%%%%%%%%%%%%%%%%%%%%%%%%%%%%%%%%%%%%%%%%%%%%%%%%%%%%%%%%%%%%%%%%%%%%%%%%%%
%\input{6 Discussion}
%%%%%%%%%%%%%%%%%%%%%%%%%%%%%%%%%%%%%%%%%%%%%%%%%%%%%%%%%%%%%%%%%%%%%%%%%%%%%%%%
%\input{7 Conclusion}
%%%%%%%%%%%%%%%%%%%%%%%%%%%%%%%%%%%%%%%%%%%%%%%%%%%%%%%%%%%%%%%%%%%%%%%%%%%%%%%%

\chapter{Introduction} \label{Chap1}

\section{Background}

The first announcement of the construction of Bogota's future metro system was on XXXXXX by XXXX. On several occasions, the need for a higher-capacity public transportation system in Colombia's capital has taken great attention from the general public. The initial idea of building a metro in Bogot\'{a} goes back to the year 1942 when the current mayor at the time suggested the building of a new metro system as the functioning trolley car was highly demanded \citep{metrodebogotaHistoriaMetroBogota2011}. The previous versions of the project were unsuccessful in terms of their execution and compilation, however, each attempt reinforced the need of improving Bogota's public transportation system.

The scale and dimension of Colombia's capital, Bogot\'{a} can be measured from multiple angles. Population-wise, with 7.9 million people, it is the country's biggest city and accounts for 15\% of the country's total population, as Medell\'{i}n and Cali take second and third place with 5\% and 4\% of the total population, respectively \citep{daneProyeccionesPoblacionPopulation2023}. Regarding the city's economic role, it is the biggest economic hub with 25\% of the nation's GDP, followed similarly as in the population distribution by Medell\'{i}n and Cali, with 6\% and 4\%, respectively \citep{daneCuentasNacionalesDepartamentales2023}.

Reference to BRT

By September 2016, the local and national governments finally materialized their intentions in a formal agreement to support the development of the metro project which led to the ongoing construction process that started in 2021. The project consists of the construction of the first of two future metro lines. The scope of this research will only include the first metro line in the future scenario as the final designs of the second metro line have not been published yet. 

The first metro line will have a 23,9-kilometre length overground track with 16 stations. With an investment of 12,95 billion COP (4,33\footnote{September 2017 average COP/USD rate: 2.991,42 \citep{bancodelarepublicaTasaCambioRepresentativa2023}} billion USD), the construction of the first line began in 2021 and is supposed to be finished by 2028.


% \begin{itemize}
%  \item Lack of functioning metro system in Bogota although it has been proposed and mention in previous local and/or national government plans
% \end{itemize}

\section{Importance}

The improvement of the current transport infrastructure system in Colombia's capitals aims to impact a significant share of Bogot\'{a}'s population, as 1 in every 3 trips in the city relies on public transportation. According to Bogot\'{a} City Council, BRT is the primary mode of commuting, with 4,8 million trips a day \citep{alcaldiadebogotaEncuestaMovilidad20192019}. From the total of 13.4 million daily trips, BRT accounts for 36\% total trips, followed by walking\footnote{Pedestrian trips with length greater or equal to 15 minutes} and privately owned cars, with 24\% and 15\% of total trips, respectively.



\begin{itemize}

\item Congested BRT Network

\item Brief intro to Bogota as a city and centre of economic and decision-making processes
\item How does it contribute or prevent people to have better life quality
\item It takes advantage of the use of publicly available data from transport and local authorities (spatial and non-spatial)
  \item Provide technical tools to assess transport planning decision
  \item Baseline reference for future or similar transport public policy decision making
  \item Assess how the metro system designs address the accessibility of the inhabitants
  \item Spatial reference for future local government intervention and investment decisions from the private sector
  \item Highlight accessibility importance in an economic and urban context and How can it drive growth
  \item Mention the impact on making cities more "liveable" and contribute to having a better life quality
\end{itemize}

\section{Research question}

The present research aims to address:

% Initial one


% \begin{center}
%     \textit{How would the public transportation accessibility spatially vary with the construction of Bogot\'{a}\textquotesingle s future metro system?}
% \end{center}

% Adjusted one

% \begin{center}
%     \textit{How would Bogot\'{a}\textquotesingle s future metro lines contribute to improving access to opportunities? To what extent is addressing the socio-spatial inequalities in Colombia\textquotesingle s capital?}
% \end{center}

% Current one

\begin{center}
    \textit{How would Bogot\'{a}\textquotesingle s future metro lines contribute to improving access to opportunities? How would the public transportation accessibility spatially vary with the construction of Bogot\'{a}\textquotesingle s future metro system?}
\end{center}



% \subsection{Subsection}
% \subsubsection{SubSubsection}
% \paragraph{Parragraph}
% \lipsum[5]
% Testing \Gls{raster}

Testing footnote\footnote{Capion of footnote!}

Testing references \citep{alcaldiamayordebogotad.c.EstacionesPrimeraLinea2022}


\chapter{Literature review} \label{Chap2}

\section{Modelling in the urban context}

The use of modelling tools in the urban environment has gained popularity for analyzing high-complexity topics to support the decision-making process \citep{houApproachBuildingOccupancy2020}. As urban theories can be represented in mathematical models, the use of simulation processes allows researchers to experiment with possible outcomes in future urban scenarios \citep{battyUrbanModeling2009a}.

Regarding the urban modelling process as a discipline both, \cite{battyUrbanModeling2009a} and \cite{wilsonFutureUrbanModelling2018}, presented past references to grasp a general notion of its historic evolution. On one hand, \cite{battyUrbanModeling2009a} stated how the first formal contributions on the topic go back to \cite{alonsoLocationLandUse1964}, the basis of the so-called new urban economics. Batty included a model-building process overview listing the main scientific and mathematics principles to consider. On the other hand, \cite{wilsonScienceCitiesRegions2012} highlighted the influence of \cite{lowryModelMetropolis1964} as a milestone in the urban modelling evolution and uses it as an example to explain urban modelling together with a full historical review of the field.

Taking \cite{battyUrbanModeling2009a} and \cite{wilsonFutureUrbanModelling2018} as references, the urban models can be grouped into three main categories: 

\begin{enumerate}
  \item Land-use transportation: Conceptually based on the gravitational model \citep{battyUrbanModellingAlgorithms1976}, it aims to predict the spatial interaction (flows) between land use, economic activity and the inherent transportation costs of spatial units within the area of study. Leveraging on the principles of Newton's model, these models forecast how spatial units would  interact with one another considering the cost of interaction and attractiveness between them.
  \item Urban dynamics: Built on the principles used in ecological dynamic models and system dynamics, the urban dynamics models intend to derive through time the aggregate future dynamic activity or state that may result within an urban system.
  \item Cellular automata and agent-based modelling: They aim to simulate the emergent aggregate result of individual agent-level actions in a system, as agents interact with both the environment and other agents.
  \item Network analysis: It allows to study of the connections patterns that may arise between components to systematically assess the collective behaviour of components as a whole \citep{newmanNetworksIntroduction2010}.
\end{enumerate}

With the emergence of computational modelling, \cite{battyUrbanModeling2009a} described how it allowed urban theories to be deployed in an experimental digital environment. As theoretical knowledge can be represented in mathematical models, the use of simulation processes allowed researchers to experiment with hypothetical circumstances and produce possible outcomes in future urban scenarios \citep{battyUrbanModeling2009a}.

% Refernce to the futuro portraid by Wilson

In a more prospective and industry-related exercise, \cite{wilsonFutureUrbanModelling2018} presented his view on how urban modelling will develop and prosper in the future. Regarding tools and the inputs to further urban modelling development, \cite{wilsonFutureUrbanModelling2018} mentioned how big data and technological advancements will enable new capacities and new input data to apply modelling in the urban context. Apart from the already relatively well-developed urban modelling tools designed for transport assessment and retail location intelligence, \cite{wilsonFutureUrbanModelling2018} listed other industries that could expand the use of spatial modelling to support their location decision-making processes. Among the high-use modelling potential situations, \cite{wilsonFutureUrbanModelling2018} mentioned the energy sector, global trading models and the defence and security industry.

% Move to other section
% Technological information developments for the last decades have enabled greater capture, storing and processing capacities \citep{krainesIntegratingDistributedComputational2011}, a more multidisciplinary and robust approach is now available when addressing built environment-related phenomena. 

Given these points, it was highlighted by \cite{battyUrbanModeling2009a} and \cite{wilsonFutureUrbanModelling2018} how these modelling tools hold great value for both policymakers and private parties, as they manage to integrate theoretical constructs or business strategies and put them to the test through computational modelling \citep{battyUrbanModeling2009a}. Either from a public, private or research perspective, modelling tools applied to the urban context could provide decision-makers with new innovative insights and possible solutions to the numerous challenges that cities and society currently face across the globe.

Furthermore, \cite{battyUrbanModeling2009a} clarified that apart from the computational models above mentioned, GIS-based models are also part of the urban modelling domain and will be described in the next section of the present document.

% Mention to the groups described by Batty, complemented with the ones described by Clarke's article




% which included the main model types, their applications and future uses for urban policy-making. 



% In order to deploy these state-of-the-art computational tools on the subject of interest, data is required. Apart from the underlying theory of the model, data that intends to capture the topic of interest is essential to leverage computational models to contribute to urban-related topics. The expansion of urban modelling is highly related to the data availability to use as input in the simulation process. Luckily, there have been enormous 

% Without input data for the modelling process, the future of urban modelling would be highly compromised, luckily that is not the case. 

% Open data

% Open government

% Data science

\section{Geographic information system}

Geographic information can be defined as the result of combining non-spatial data with their location in space. \cite{longleyGeographicInformationScience2015} defined geographic information as data that represents the 'what' and the 'where' features of anything. The 'what' component of the data would account for the attributes or non-spatial data captured and the 'when' is responsible for holding the coordinates that symbolise a specific location on earth.

According to \cite{maclachlanAppliedGeographicInformation2022} geographic information can be divided into two main data types:

\begin{enumerate}
  \item Vector: Geographic vector data typically includes points, lines, polygons and grids.
  \item Raster: The raster format consists of a set of cell grids in which each cell holds a specific value. 
\end{enumerate}

To illustrate the use of geographic data, figure \ref{fig:Fig_data_types} shows the fashion in which data type can deliver a fair representation of the real world, through the use of both vector and raster data.

\begin{figure}[htp]
    \centering
    \includegraphics[width=6cm]{Images/Fig_Types_Geodata.png}
    \caption{Vector and raster geographical representation \citep{saabConceptualizingSpaceMapping2003}}
    \label{fig:Fig_data_types}
\end{figure}


% Include image of types of data: Look in \cite{longleyGeographicInformationScience2015} or use image from Andys book

The main use of geographic information is to represent the relative locations of features through a schematic and visual geographic model. With a certain level of abstraction, detail and scale, geographic modelling (mapping) produces knowledge related to the location and the features of the subject of study \citep{longleyGeographicInformationScience2015}. Hence, a geographic information system (GIS), consists of the systematic collection of geo-referenced attribute data. Gifted with powerful processing capabilities, 'GI systems are computer-based tools for collecting, storing, processing, analyzing, and visualizing geographic information' \citep{longleyGeographicInformationScience2015}.

% Geographic model

% enables one to record and represent the location  

% Abstraction and scientific method reference

% Map as a model to represent reality

% Reference to urban modelling for transport

As computing does in most cases, computer-based GIS offers the capacity of processing data in a fraction of the time that would take if is performed manually. Similarly, the generation and capture capacity of location-related attributes has widespread \citep{longleyGeographicInformationScience2015}, as the use of digital means has expanded to support all kinds of services and activities. The conjunctions of these two circumstances are aligned with the future scenario portrayed by \cite{wilsonFutureUrbanModelling2018} with the use of urban modelling covering new unattended spatial modelling and analysis need.



In that sense, the location of activities, infrastructure, points of interest or any kind of subject that includes location attributes can serve as input in the urban modelling process to test location-sensitive scenarios. 

% REfenrece to specific GIS measures

% Example of GIS urban modelling

\section{Accessibility modelling}

\begin{itemize}
  \item Reference to Global UN Sustainability goals related to urban development and livable cities
  \item Reference to initiatives and efforts from NGO organizations related to urban accessibility: IBD, World Bank, GIZ.
  \item Reference to north and global south previous research
  \item Reference to Rafael Pereira's work
\end{itemize}

\section{Colombia and the Bogot\'{a} study case}

\begin{itemize}
  \item Reference to National (Colombia) and local government (Bogota) plans related to transport infrastructure (or in Chapter 3?)
  \item Reference to research done in Bogota (or in Chapter 3?)
\end{itemize}

\chapter{Study area and or data chapter} \label{Chap3}

\section{Bogot\'{a} summary (Should I do this in the introduction?)}

\begin{itemize}
  \item Brief context of Bogot\'{a} as Colombia\textquotesingle s capital
  \item Current Bogot\'{a}\textquotesingle s spatial structure
  \item Social and spatial population distribution
\end{itemize}

\section{Data}

\subsection{Population and demographics}

\subsection{Land use}

\subsection{Transport infrastructure}

\subsubsection{Bus Rapit Transit network}

\subsubsection{Metro network}



\chapter{Methodology} \label{Chap4}

\section{Accessibility measures}

\subsection{Cumulative accessibility}

\subsection{Minimum time to opportunities}

\section{Accessibility modelling}

\chapter{Results} \label{Chap5}
\chapter{Discussion} \label{Chap6}
\chapter{Conclusion} \label{Chap7}


\renewcommand{\bibname}{References}
\bibliographystyle{agsm}
%\bibliography{Bibliography.bib}
\bibliography{references.bib}
%%%%%%%%%%%%%%%%%%%%%%%%%%%%%%%%%%%%%%%%%%%%%%%%%%%%%%%%%%%%%%%%%%%%%%%%%%%%%%%%
% APPENDIX
\begin{appendices}
\chapter{Source code and data} \label{System Requirements}
Source code and data for all of the methods implemented in Chap. \ref{Chap4} for the project can be found in the remote repository: \href{https://github.com/rpoandres/MSc_USS_Dissertation}{GitHub}



% \chapter{Project Introduction Video}\label{sec:projectIntroduction}
% A short video presentation, introducing background, aims and organisation of the project, as of 30$^{th}$ June 2022: \newline
% \url{link}



\end{appendices}
%%%%%%%%%%%%%%%%%%%%%%%%%%%%%%%%%%%%%%%%%%%%%%%%%%%%%%%%%%%%%%%%%%%%%%%%%%%%%%%%
% GLOSSARY
\clearpage
\printglossaries

% INDEX?

\end{document}